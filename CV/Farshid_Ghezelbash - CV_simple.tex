\documentclass[11pt,a4paper]{article}

% Packages
\usepackage[utf8]{inputenc}
\usepackage[T1]{fontenc}
\usepackage{geometry}
\usepackage{hyperref}
\usepackage{lmodern}
\usepackage{enumitem}
\usepackage{setspace}

% Page layout
\geometry{
  left=1in,
  right=1in,
  top=1in,
  bottom=1in,
}

% Hyperlink setup
\hypersetup{
  colorlinks=true,
  urlcolor=blue,
  linkcolor=black,
  pdfauthor={Farshid Ghezelbash},
  pdftitle={Curriculum Vitae},
}

% Section formatting
\usepackage{titlesec}
\titleformat{\section}{\large\bfseries}{\thesection}{1em}{}
\titlespacing*{\section}{0pt}{1.5ex plus 1ex minus .2ex}{1ex plus .2ex}

% Remove indentation
\setlength{\parindent}{0pt}
\setlength{\parskip}{0.5em}

% Begin document
\begin{document}

% Name and contact information
\begin{center}
    {\LARGE \textbf{Farshid Ghezelbash, PhD}}\\
    \vspace{1ex}
    \href{mailto:ghezelbash.far@gmail.com}{ghezelbash.far@gmail.com} \\
    \href{https://www.linkedin.com/in/fred-farshid-ghezelbash-818258206/}{LinkedIn} \textbullet{} 
    \href{https://scholar.google.com/citations?user=Ryg1OQgAAAAJ&hl=en}{Google Scholar} \textbullet{}
    \href{https://sites.google.com/view/farshidghezelbash/home}{Personal Webpage} \\
    Phone: +1 (438) 992-3099 \\
\end{center}

% Statement
\section*{Statement}
With a vision to integrate advanced research with real-world product development, I aim to create innovative technologies and products. I have developed multiple computational models that are being used by various clinical and industrial partners, demonstrating my ability to translate research into real-world applications.

% Summary
\section*{Summary}
\begin{itemize}[leftmargin=*, label={}]
    \item \textbf{Expertise:} Extensive background in mechanical engineering, computational mechanics, and biomechanics.
    \item \textbf{Innovation:} Proven track record in leading cross-functional R\&D projects from concept to mass production, with an emphasis on computational methods, data analysis, and user experience.
    \item \textbf{Publications \& Patents:} Filed one patent and published 28 journal papers (majority on computational mechanics).
    \item \textbf{Recognition:} Recipient of multiple prestigious scholarships and fellowships.
\end{itemize}

% Selected Experience
\section*{Selected Experience}

\textbf{Biomechanical Scientist \hfill 2022--Present} \\
EERS, QC, Canada
\begin{itemize}[leftmargin=*, label={}]
    \item Led R\&D projects focusing on the development of ergonomic auditory wearables.
    \item Conducted finite element analysis (FEA) of designed auditory wearables for performance and comfort.
    \item Applied machine learning and data-driven approaches to product design.
    \item Utilized computational design and shape analysis to enhance product ergonomics and mass manufacturing.
    \item Directed large-scale trials to assess the comfort, usability, and acoustic performance of auditory wearables.
    \item Conducted advanced material research and mechanical characterization.
\end{itemize}

\textbf{Postdoctoral Scholar \hfill 2020--2022} \\
McGill University, QC, Canada
\begin{itemize}[leftmargin=*, label={}]
    \item Pioneered damage and fracture characterization of intervertebral discs.
    \item Developed finite element models to capture mechanical response of intervertebral discs and blood clots.
    \item Explored novel mechanical properties of blood clots.
    \item Led spine biomechanics team and supervised graduate students.
    \item Awarded beamline time at CLS (the only synchrotron facility in Canada).
\end{itemize}

\textbf{Postdoctoral Researcher \hfill 2020--2022} \\
IRSST \& Polytechnique Montreal, QC, Canada
\begin{itemize}[leftmargin=*, label={}]
    \item Developed the first micromechanical finite element model of the human disc.
    \item Created an advanced EMG-driven finite element model for patients with back pain.
    \item Employed machine learning for biomechanical assessment and advanced modeling.
    \item Conducted biomechanical experiments (electromyography \& kinematics).
    \item Wrote research grants (IRSST and NSERC).
\end{itemize}

\textbf{Research Assistant \hfill 2015--2019} \\
Polytechnique Montreal, QC, Canada
\begin{itemize}[leftmargin=*, label={}]
    \item Developed and validated a subject-specific musculoskeletal model.
    \item Constructed a finite element model of the intervertebral disc to capture damage and failure.
    \item Established an analytical tool to estimate spinal loads for an industrial partner.
    \item Collected and analyzed experimental data (whole-body kinematics \& electromyography).
    \item Wrote research grants (FRQNT).
\end{itemize}

% Skills
\section*{Skills}

\begin{itemize}[leftmargin=*, label={}]
    \item \textbf{Theoretical Skills:} Computational mechanics, finite element analysis (FEA), constitutive modeling, solid mechanics, musculoskeletal modeling, machine learning, data science
    \item \textbf{Modeling \& Simulation:} Abaqus, COMSOL, HyperMesh, OpenSim, 3DSSPP
    \item \textbf{Programming:} Python, MATLAB, Fortran, C\#
    \item \textbf{Experimental Skills:} Biomechanical human trials, comfort tests, motion tracking, mechanical material testing (tension, compression), microscopy, histopathology
    \item \textbf{3D Design:} SolidWorks, Rhino, Grasshopper
    \item \textbf{Prototyping:} 3D printing, laser cutting, micro-controllers
    \item \textbf{Collaboration:} Worked with teams from embedded systems, clinicians, manufacturing, and materials science to achieve project goals
\end{itemize}

% Honors and Awards
\section*{Honors and Awards}

\begin{itemize}[leftmargin=*, label={}]
    \item Postdoctoral fellowship, FRQNT, \$90K, Canada (2023)
    \item Postdoctoral fellowship, Mitacs Elevate, \$65K, Canada (2020--2022)
    \item Awarded beamline time at CLS (the only synchrotron facility in Canada) (2021)
    \item Postdoctoral fellowship, Merit Scholarship, FRQNT, \$9K, Canada (2020)
    \item Doctoral scholarship, Merit Scholarship, FRQNT, \$49K, Canada (2016--2018)
    \item Ranked 4th among 24,000 candidates, National MSc Entrance Exam, Iran (2012)
    \item Top student, Mechanical Engineering Department, K.N.T.U., Iran (2011)
\end{itemize}

% Publications
\section*{Publications}

\begin{itemize}[leftmargin=*, label={}]
    \item Filed one patent on computational design of auditory wearables, and two design patents.
    \item 28 published papers in scientific journals.
    \item 16 presentations at scientific conferences.
    \item Invited speaker at various workshops.
    \item Full list of publications available on \href{https://scholar.google.com/citations?user=Ryg1OQgAAAAJ&hl=en}{Google Scholar} (citations: 585; h-index: 14).
\end{itemize}

% Selected Projects
\section*{Selected Projects}

\begin{itemize}[leftmargin=*, label={}]
    \item \textbf{2022--2024:} Created a data-driven pipeline to design ergonomic auditory wearables.
    \item \textbf{2021--2024:} Developed a machine learning framework to evaluate spine biomechanics.
    \item \textbf{2021--2023:} Constructed a complex finite element model of the spine with clinical applications.
    \item More details are available on my \href{https://sites.google.com/view/farshidghezelbash/home}{personal website}.
\end{itemize}

% Education
\section*{Education}

\textbf{MSc in Computer Science \hfill 2024--Present} \\
Georgia Institute of Technology

\textbf{PhD in Mechanical Engineering \hfill 2015--2019} \\
Polytechnique Montreal, Canada

\textbf{MSc in Mechanical Engineering \hfill 2012--2014} \\
Sharif University of Technology, Tehran, Iran

\textbf{BSc (Hons) in Mechanical Engineering \hfill 2007--2011} \\
K.N.Toosi University of Technology, Tehran, Iran

% References
\section*{References}
Available upon request.

\end{document}
